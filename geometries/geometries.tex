\documentclass{article}

\usepackage{bbm}
\usepackage{amsmath}
\usepackage{amssymb}
\title{Géométries}
\author{William Hergès}
\date{\today}

\begin{document}
	\maketitle

	\section{Introduction}

	La géométrie dans le plan et dans l'espace fonctionne de la même manière au niveau des notations à l'unique différence que dans l'espace il y a 3 coordonnées et non plus que 2.

	\section{Appartenance à une droite et plan}

	Soit $M$ un point de $(AB)$.
	\[M \in (AB) \] \begin{center} $\Leftrightarrow \vec{AM}$ colinéaire à $\vec{AB}$ \end{center}
	\[\Leftrightarrow \vec{AM} \Leftrightarrow k\vec{AB} ; k \in \mathbb{R}\]

	Soit $A,B,C$ des points non alignés du plan $(ABC)$. \\
	Soit $M$ un point du plan $(ABC)$.
	\[
		M \in (ABC) \Leftrightarrow \vec{AM} = x\vec{AC} + y\vec{AB} ; x,y \in \mathbb{R}
	\]
	Soit $N$ un point du même plan. \\
	$M$ et $N$ sont coplanaires. \\
	\\
	$\vec{AB}$ et $\vec{AC}$ sont des vecteurs directeurs du plan $(ABC)$. $(\vec{AB}, \vec{AC})$ est une base de ce plan et $A, \vec{AB}, \vec{AC}$ un repère de ce plan.

	Pour calculer la norme d'un vecteur, il faut connaître les coordonnées de ce dernier, ici noté $x;y;z$.
	\[ |AB| = \sqrt{x^2+y^2+z^3} \]

	Si on souhaite connaître la distance entre le point $A$ et le plan $P$ de vecteur normal $\vec{n}$, il nous faut un point $B \in P$. \\
	Cette distance est représenté par $AH$ avec $H$ le projeté orthogonal de $A$ sur $P$.
	\[ AH = \frac{|\vec{AB} \cdot \vec{n}|}{\|\vec{u}\|} \]

	Si on souhaite connaître la distance entre le point $A$ et la droite $d$ de vecteur directeur $\vec{u}$, il nous faut un point $B \in d$. \\
	Cette distance est représenté par $AH$ avec $H$ le projeté orthogonal de $A$ sur $P$.
	\[ AH = \|\vec{AB}-\frac{\vec{AB}\cdot\vec{u}}{\|\vec{u}^2\|}\vec{u}\| \]

	\section{Théorème du toit}

	Soient $P_1$ et $P_2$ deux plans sécants suivant une droite $d$. \\
	\\
	Si une droite est parallèle à $P_1$ et $P_2$, alors elle est parallèle à $d$. \\
	S'il existe deux droites parallèles $d_1$ et $d_2$, contenues respectivement dans $P_1$ et $P_2$, alors $d$ est parallèle à $d_1$ et $d_2$.

	\section{Produits scalaires}

	Rappels, un produit scalaire entre $\vec{u}$ et $\vec{v}$ se note :
	\[ \vec{u} \cdot \vec{v} \]

	Un produit scalaire dans une base orthonormée de l'espace se calcule 
	\[ \vec{u} \cdot \vec{v} = xx'+yy'+zz' \]
	
	$\vec{u}$ et $\vec{v}$ sont orthogonaux si et seulement si $\vec{u} \cdot \vec{v} = 0$.

	\section{Représentation paramétrique}

	Soit un point $A$ tel que $A (x_a,y_a,z_a)$ et $\vec(u) \begin{pmatrix}x_u\\y_u\\z_u\end{pmatrix}$. \\
	Soit $d$ la droite passant par $A$ de vecteur directeur $\vec{u}$. \\
	Il existe une représentation paramétrique de cette droite de forme :
	\[ 
		\left\{ 
			\begin{array}{cc}
				x = & t*x_u + x_a \\
				y = & t*y_u + y_a \\
				z = & t*z_u + z_a
			\end{array}
		\right.
		; t \in \mathbb{R}
	\]
	Cette représentation n'est pas unique : il en existe une infinité. \\
	\\
	Soit un point $M(x_m,y_m,z_m)$. \\
	\[ M \in d \Leftrightarrow M \in d(A;\vec{u}) \Leftrightarrow \vec{AM}=t\vec{u};t\in\mathbb{R}\]
	Cela revient à résoudre le système d'équation qu'est la représentation paramétrique de $d$ avec $x=x_m;y=y_m;z=z_m$.

	\section{Équation cartésienne de plan}

	Soit $A(\alpha,\beta,\gamma)$ et le vecteur normal $\vec{n} \begin{pmatrix}a\\b\\c\end{pmatrix}$ qui forment le plan $P$. \\
	Soit $M(x,y,z)$.
	\[
		M\in P \Leftrightarrow ax+by+cz-(\alpha*a+\beta*b+\gamma*c) = 0 \Leftrightarrow \vec{AM} \cdot \vec{n} = 0
	\]

	Une équation cartésienne de plan est donc de forme $ax+by+cz-(\alpha*a+\beta*b+\gamma*c) = 0$ avec $(x,y,z)\in\mathbb{R}^3$ \\
	Tous points vérifiant cette équation font partie du plan.
\end{document}

