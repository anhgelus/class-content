\documentclass{article}

\usepackage{bbm}
\usepackage{amsmath}
\usepackage{amssymb}
\title{Probabilités}
\author{William Hergès}
\date{\today}

\begin{document}
	\maketitle

	\section{Ensemble}

	L'ensemble vide est inclu dans $\mathbb{R}$. \\
	\\
	Le nombre total de sous-élément de $E$, un ensemble à $n$ éléments, est égale à :
	\[ 2^n\]
	Soit $n$ et $k$ tel que $0 \leqslant k \leqslant n$. \\
	Le nombre de combinaison de $k$ éléments d'un ensemble à $n$ éléments est noté $ 
		\begin{pmatrix}
                	n \\
                        k
		\end{pmatrix}
	$, et est donné par $
		\begin{pmatrix}
                	n \\
                        k
		\end{pmatrix} = \frac{n!}{(n-k)!k!}
	$.

	\section{Épreuves indépendantes}

	Soit une succession de n épreuves indépendantes. \\
	L'univers des issues possibles est le produit cartésien $\Omega_1 * \Omega_2 * ... * \Omega_n$. \\
	Soit une issue $(i_1,i_2,i_3,...,i_n)$, sa propabilité est le produit des probabilité de chacune des issues du n-uplet.

	\section{Loi Binominale}

	Une épreuve de Bernouilli est une épreuve à issue $S$ de probabilité $P$ et $\overline{S}$ de probabilité $1-P$. \\
	Le schéma de Bernouilli est une succession d'épreuve de Bernouilli identique et indépendante. \\
	\\
	Soit $n\in\mathbb{N*}$ le nombre de succession d'épreuve lors d'un schéma de Bernouilli et $X$ le nombre de succès. \\
	La probabilité que $X$ arrive $k$ fois est donnée par :
	\[ P(X = k) = \begin{pmatrix}n \\ k\end{pmatrix}*P^k*(1-P)^{n-k} \]
	Dans ce cas là, on dit que $X$ suit la loi binominale de paramètre $n$ et $P$. \\
	L'espèrence de $X$, noté $E(X)$, est $E(x) = np$. La variance de $X$, noté $V(x)$, est $V(x) = np(1-p)$. L'écart-type de $X$, noté $\sigma(X)$, est $\sigma(X) = \sqrt{np(1-p)}$
\end{document}

