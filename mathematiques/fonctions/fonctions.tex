\documentclass{article}
\usepackage{bbm}
\usepackage{amsmath}
\usepackage{amssymb}
\title{Fonctions}
\author{William Hergès}
\date{\today}

\begin{document}
	\maketitle

	\section{Introduction}

	Les fonctions peuvent être dérivées pour aider leurs études. La dérivation permet notamment d'étudier la convexité ainsi que la varation de la fonction. \\
	En effet, la dérivé première, noté $f'$ pour la fonction $f$, permet d'obtenir la variation de la fonction, c'est-à-dire si elle est croissante ou décroissante. \\
	\\
	Une fonction est dérivable s'il s'agit d'un polynôme, ou d'une somme/produit/quotient de fonction dérivable sur $I$. De plus, les fonctions $\cos$ et $\sin$ sont dérivables sur $\mathbb{R}$

	\section{Dérivée de fonction simple}
	Pour dérivée une fonction, on utilise ce tableau suivant dans un grand nombre de cas :
	
	\begin{center}
		\begin{tabular}{|c|c|}
	                \hline
		        Base & Dérivée \\
		        \hline
			$x^n$ pour $n \in \mathbb{N}$ & $nx^{n-1}$ \\
		        \hline
			$e^{x}$ & $e^{x}$ \\
		        \hline
			$\ln x$ & $x^{-1}$ \\
			\hline
			$\sin x$ & $\cos$ \\
			\hline
			$\cos$ & $\sin$ \\
			\hline
	        \end{tabular}
	\end{center}

	\section{Dérivées de produit ou de quotient de fonctions}

	Les dérivées de produit ou de quotient ne suivent pas les règles plus haut car on ne dérive pas le contenu de la fonction mais la fonction. \\
	\\
	Nous allons d'abord voir les cas généraux :

	\begin{center}
		 \begin{tabular}{|c|c|}
			 \hline
			 Forme de Base & Dérivée \\
			 \hline
			 $u(x)*v(x)$ & $u'(x)*v(x)+u(x)*v'(x)$ \\
			 \hline
			 $\frac{u(x)}{v(x)}$ & $\frac{u'(x)*v(x)-u(x)*v'(x)}{(v(x))^2}$ \\
			 \hline
		 \end{tabular}
	\end{center}

	\section{Dérivées de composition de fonctions}

	Comme pour les dérivées de produit et de quotient, les dérivées de composition de fonctions ne se dérive pas de la même manière que les autres. \\
	Ici, à l'instar des autres méthodes, il existe quelques formules spécifiques qui permettent de dérivée le tout beaucoup plus rapidement. \\
	\\
	Commençons d'abord par voir le cas général :
	\begin{center}
		 \begin{tabular}{|c|c|}
			 \hline
			 Forme de Base & Dérivée \\
			 \hline
			 $u(x)\circ v(x)$ ou $u(v(x))$ & $v'(x)*v(u'(x))$ \\
			 \hline
		 \end{tabular}
	\end{center}
	Ensuite, pour aller plus rapidement, nous pouvons retenir des formes spécifiques. \\
	\begin{center}
		 \begin{tabular}{|c|c|}
			 \hline
			 Forme de Base & Dérivée \\
			 \hline
			 $e^{u(x)}$ & $u'(x)*e^{u(x)}$ \\
			 \hline
			 $\ln (u(x))$ & $\frac{u'(x)}{u(x)}$ \\
			 \hline
			 $\sqrt{u(x)}$ & $\frac{u'(x)}{2\sqrt{u(x)}}$ \\
			 \hline
			 $(u(x))^n$ avec $n \in \mathbb{N}$ & $nu'(x)(u(x))^{n-1}$ \\
			 \hline
		 \end{tabular}
	\end{center}

	L'ensemble de définition de ces fonctions composées sont l'ensemble des réels x appartenant à $D_v$ dont l'image par $v$ appartient à $D_u$.

	\section{Propriété de la dérivation}

	La fonction dérivée et sa primitive sont liées. \\
	En effet, la varition de la primitive dépend du signe de sa dérivée. Si sa dérivée est négative sur $I$, alors elle est décroissante sur $I$, et si sa dérivée est positive sur $I'$, alors elle est croissante sur $I'$. \\
	\\
	De plus, il existe aussi un lien entre la primitive et sa dérivée seconde. \\
	La convexité dépend de la primitive dépend de sa dérivée seconde. Si sa dérivée seconde est négative sur $I$, alors elle est concave et sa dérivée est décroissante sur $I$, et si sa dérivée seconde est positive sur $I'$, alors elle est convexe et sa dérivée est croissante sur $I'$.

	\section{Primitive}

	La primitive est la méthode inverse de la dérivation. \\
	Elle permet de passer de $f'$ à $f$. \\
	\\
	Pour prendre la primitive d'une fonction, il faut utiliser les formules que nous avons vu lors plus haut.

	\begin{center}
		\begin{tabular}{|c|c|}
	                \hline
		        Dérivée & Primitive \\
			\hline
			$x^n$ & $\frac{x^{n+1}}{x^{n+1}}$ \\
			\hline
			$\frac{1}{x}$ & $\ln x$ \\
			\hline
			$\frac{1}{\sqrt{x}}$ & $2\sqrt{x}$ \\
			\hline
			$e^x$ & $e^x$ \\
			\hline
			$\cos x$ & $\sin x$ \\
			\hline
			$\sin x$ & $-\cos x$ \\
			\hline
		\end{tabular}
	\end{center}

	\section{Équation de la tangeante}

	Équation de la tangeante en un point d'abscisse A avec $f(x)$ la fonction :
	\[ y=f'(a)(x-a)+f(a) \]
	Mémotechnique : convexe = forme un V, concave forme une cave.
	
	\section{Continuité}

	Soit $a\in I$. \\
	$f$ est continue en a quand $\lim_{x \to a} = \lim_{x\to -a} = f(a)$. \\
	Les sommes, produits, quotients et compositions de fonctions continues donnent une fonction continue sur son ensemble de définition. \\
	Si $f$ est dérivable sur $I$, alors elle est continue sur $I$.

	\section{Théorème des valeurs intermédiaires}

	Si $f$ est continue et strictement croissante, que $f(a) = \alpha$, que $f(b) = \beta$ et que $\alpha < \beta$, avec $a,b,\alpha,\beta \in \mathbb{R}$ et $a<b$ ; ou si $f$ est continue et strictement décroissante, que $f(a) = \alpha$, que $f(b) = \beta$ et que $\alpha < \beta$, avec $a,b,\alpha,\beta \in \mathbb{R}$ et $b<a$, alors
	\[ \exists l \in \mathbb{R} / \alpha < l < \beta \]

	\section{Théorème}

	Soit $f$, une fonction continue sur $\mathbb{R}$ et $u_n$ une suite à valeur sur $\mathbb{R}$ défini par $u_{n+1} = f(u_n)$. \\
	Si $u_n$ converge vers $l$ si et seulement si $f(x)$ converge vers $l$. 
	\[ \lim f(x) = \lim u_{n+1} = \lim u_n = l \]
	Ainsi :
	\[ f(l) = l \]
	Car $f(x) = f(l)$ car $x$ est quelconque.


	\section{Logarithme népérien}

	La fonction $\ln x$ est la fonction réciproque de $e^x$. \\
	\begin{center}
		\begin{tabular}{|c|c|}
			\hline
			Forme & Simplification \\
			\hline
			$\ln (a*b)$ & $\ln a + \ln b$ \\
			\hline
			$\ln (\frac{a}{b})$ & $\ln a - \ln b$ \\
			\hline
			$\ln (a^n)$ pour $n\in\mathbb{Z}$ & $n\ln a$ \\
			\hline
		\end{tabular}
	\end{center}
	$\ln$ est défini sur $\mathbb{R}^{+*}$. Sa dérivé est $x^{-1}$. \\
	\\
	Quelques propriétés et notations découlant de $\ln$ :
	\[
		\log x = \frac{\ln x}{\ln 10}
	\]
	\[
		\log_a x = \frac{\ln x}{\ln a}
	\]
	La fonction $\log$ permet est la réciproque de $10^x$ et la fonction $\log_a$ est la réciproque de $a^x$.

	\section{Limites des logarithmes}

	La fonction $\ln$ et ses dérivées possèdent de nombreuses particularités au niveau des limites. \\
	Quand on applique ces propriétés, on indique "par croissance comparé". \\
	\[
		\lim_{x \to 0} x^n*\ln x = 0;n\in\mathbb{Z}
	\]
	\[
		\lim_{x \to +\infty} \frac{\ln x}{x^n} = 0;n\in\mathbb{Z}
	\]
\end{document}

