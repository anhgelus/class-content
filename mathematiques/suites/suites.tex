\documentclass{article}

\usepackage{bbm}
\usepackage{amsmath}
\usepackage{amssymb}
\title{Suites}
\author{William Hergès}
\date{\today}

\begin{document}
	\maketitle

	\section{Introduction}

	Une démonstration par récurrence est efficace pour l'ensemble de démonstration se basant sur le principe d'hérédité, c'est-à-dire que si $u_k$ est vraie, alors $u_{k+1}$ est aussi vraie. \\
	Ce type de démonstration est efficace quand on a affaire avec des suites définies par récurrence, c'est-à-dire définie comme cet exemple :
	\[ \left\{ \begin{array}{l}
			u_{n+1} = u_n+4 \\
			u_0 = 2
        	\end{array}
		\right.  \]

	\section{Préparer la démonstration}

	Pour démontrer par récurrence, il faut avant tout définir la propriété mathématique que l'on souhaite démontrer. \\
	Cette propriété sera de forme :
	\[ P(n): a+b = 1 \]
	au lycée, ou de forme :
	\[ P_n: a+b = 1 \]
	dans le supérieur. \\
	Cette propriété est bien sûr une propriété booléenne, c'est-à-dire qu'elle est soit vraie, soit fausse.

	\section{Étapes d'une démonstration par récurrence}

	Une démonstration par récurrence se fait en 2 étapes si on omet l'introduction et la conclusion. \\
	\\
	La première étape est la phase d'initialisation. \\
	Cette étape permet de dire qu'il existe au moins une valeur de la suite $u_n$ tel que $P_n$ soit vraie.\\
	\\
	Ensuite, nous devons démontrer que si $u_n$ est vrai, alors $u_{n+1}$ est aussi vrai. Il s'agit de la phase d'héréditée.

	\section{Rédaction}

	Dans cette section, nous allons voir comment rédiger une rédaction par récurrence. \\
	\\
	Montrons par récurrence :
	\[ \forall n \in \mathbb{N} ; u_n = u_{n+1} - 5 \]
	Pour $n$ dans $\mathbb{N}$, on note $P(n)$ la propriété :
	\[ u_n = u_{n+1} \]
	Initialisation. On a $u_0 = 5$ et $u_1 = 0$. Or $0 = 5-5$. \\
	$P(0)$ est donc vraie. \\
	\\
	Hérédité. Fixons $n$ dans $\mathbb{N}$ tel que $P(n)$ soit vraie. On a donc : \\
	\textit{
		Ici il faut passer de $u_n$ à $u_{n+1}$ pour finir l'hérédité. \\
		Ici nous n'allons pas le faire car il ne s'agit que d'un exemple de rédaction.
	}
	C'est exactement $P(n+1)$. Ainsi $P(n)$ est héréditaire. \\
	\\
	Comme $P(0)$ est vraie et que $P(n)$ est héréditaire, par principe de récurrence, $P(n)$ est vraie pour tout $n \in \mathbb{N}$.
\end{document}

